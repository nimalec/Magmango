\documentclass{article}
\usepackage[utf8]{inputenc}
\usepackage{geometry}
\usepackage{enumitem}
\usepackage{hyperref}
\usepackage{graphicx}
\geometry{margin=1in}
\renewcommand{\baselinestretch}{1.0}
\usepackage[english]{babel}
\usepackage[utf8]{inputenc}
\usepackage{csquotes, ellipsis}
\usepackage[authordate-trad,backend=biber]{biblatex-chicago}
\usepackage{ragged2e}
\usepackage{amsmath}
\bibliography{qualrdbib.bib}
\usepackage{outlines}
\title{Magmango: the magnetic properties package}
\author{Nima Leclerc} 
\date{\today}
\raggedright

\begin{document}
\maketitle
\begin{abstract}
\justifying
We utilize tensor field networks to learn elasticity tensors from space group information for different materials in the 
\\
\\
\textbf{Keywords}: Tensor Field Networks, Elasticity, Equivariance, and Point Convolutions
\end{abstract}

\section*{Introduction}
\justifying
Convolution neural networks (CNNs) afford \textit{translational equivariance}, which is important in detecting 

\section*{Statement of Need}
\justifying


\section*{Concept}
\justifying



\section*{Constructing the spin-mesh}
\justifying


\section*{Magnetocrystalline anisotropy predictions} 
\justifying



\section*{Analysis and fitting} 
\justifying



\end{document}